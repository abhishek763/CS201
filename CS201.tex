\documentclass[12pt]{beamer}

\usetheme{Boadilla}


\title{Introduction to Combinatorial Game Theory}
\author{Abhishek Kumar\\ Manish Kumar Bera}
\institute{IITK}

\usepackage{amssymb}

\begin{document}

\begin{frame}
\titlepage
\end{frame}

\begin{frame}
\frametitle{Outline}
\tableofcontents
\end{frame}

\section{Few Combinatorial Games}

\subsection{Basic terminologies and Strategies}

\begin{frame}

\frametitle{Introduction}

\begin{definition}
Combinatorial games are two-player games with no hidden information and no chance elements. We will denote two players by \textit{L} and \textit{R}.
\end{definition}

\pause

We will implicitly assume that all games are \textit{Short}.
\pause

\begin{definition}
For any game position G we denote left options of game by $ \mathbb{g^\textit{l}}$ and right options of game by $ \mathbb{g^\textit{r}}$.\\
Thus any game position can be written as
$$G = \left\{\mathbb{g^l} | \mathbb{g^r} \right\}$$
\end{definition}

\begin{example}

\end{example}
\end{frame}
\subsection{Tic-Tac-Toe}
\begin{frame}
\frametitle{my}
my
\end{frame}
\subsection{Hex}
\frametitle{name}
\begin{frame}
name
\end{frame}
\subsection{Nim}
\title{is}
\begin{frame}
is
\end{frame}
\section{Formal Approach to Games}

\subsection{Definitions}
\begin{frame}
\frametitle{nothing}
nothing
\end{frame}

\subsection{$ \mathbb{G}$ \,as a Group}
\begin{frame}
\frametitle{like}
like
\end{frame}

\subsection{Partial Order Structure}
\begin{frame}
\frametitle{anything}
anything
\end{frame}




\end{document}